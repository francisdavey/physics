\documentclass[main.tex]{subfiles}
\begin{document}

\section{The Circle}
\section{Circular Functions (trigonometry)}
\begin{figure}[H]
  \label{fig:rfmtrigrecall}
  \begin{tikzpicture}
    \coordinate (O) at (0,0);
    \coordinate (P) at (3.2, 2.4); %(1.6, 1.2);
    \coordinate (hhtop) at (6, 2.4);
    \coordinate (A) at (4, 0);
    \coordinate (B) at (0, 4);
    \coordinate (C) at (-4, 0);
    \coordinate (D) at (0, -4);

    \fill (O) circle [radius=3pt];
    \fill (P) circle [radius=3pt];
    \fill (A) circle [radius=3pt];
    
    \draw (-6,0) -- (C) -- (O) node [below left]{O} -- (A) node [above right] {A} -- (6,0);
    \draw [<->] (6,0) -- node [right] {$R\sin\theta$}(hhtop);
    \draw (O) -- node [above] {$R$} (P) node [above right]{P};
    \draw [dotted] (P) -- (hhtop);
    \draw (O) circle (4);
    \draw [dotted] (O) -- (0, -6);
    \draw [dotted] (A) -- (4, -6);
    \draw [<->] (0, -6) -- node [below] {$R\cos\theta$} (4, -6);

    \pic [draw, ->, "$\theta$", ultra thick, angle eccentricity=1.5, angle radius=1.5cm] {angle = A--O--P};

  \end{tikzpicture}
  \caption{The definition of sine and cosine}
\end{figure}

Trigonometric functions may be written with or without parentheses $sin\theta$, $sin(\theta)$ and so on.

\section{The Ellipse}
\begin{figure}
  \label{fig:ellipsedim}
  \begin{tikzpicture}
    % Center
    \coordinate (pointC) at (0, 0);
    \fill (pointC) circle [radius=2pt];
    \draw (pointC) node [below left] {C};

    % Foci
    \coordinate (pointF1) at (4, 0);
    \fill (pointF1) circle [radius=2pt];
    \draw (pointF1) node [below] {$F_1$};
    \coordinate (pointF2) at (-4, 0);
    \fill (pointF2) circle [radius=2pt];
    \draw (pointF2) node [below] {$F_2$};

    % Vertex
    \coordinate (pointV1) at (5, 0);
    \fill (pointV1) circle [radius=2pt];
    \draw (pointV1) node [right] {$V_1$};
    \coordinate (pointV2) at (-5, 0);
    \fill (pointV2) circle [radius=2pt];
    \draw (pointV2) node [right] {$V_2$};
    \coordinate (pointV3) at (0,3);
    \fill (pointV3) circle [radius=2pt];
    \draw (pointV3) node [right] {$V_3$};
    \coordinate (pointV4) at (0,-3);
    \fill (pointV4) circle [radius=2pt];
    \draw (pointV4) node [right] {$V_4$};


    % Ellipse
    \draw (pointC) ellipse (5 and 3);


    % Axes
    \draw (pointV1) {[dotted] -- (pointF1)} -- (pointC) -- (0, 3);
    \draw (pointV2) [dotted] -- (pointC) -- (pointV4);

 
   % Axis label guidepoints
    \coordinate (label1L) at (0, -3.5);
    \coordinate (label1R) at (5, -3.5);

    \coordinate (label2L) at (-5.5, 3);
    \coordinate (label2R) at (-5.5, 0);

   % Axis labels
    \draw (pointV4) [dotted] -- (label1L);
    \draw (pointV1) [dotted] -- (label1R);
    \draw [<->] (label1L) -- (label1R) node [midway, below] (TextNode) {semi-major axis}; 

    \draw (pointV3) [dotted] -- (label2L);
    \draw (pointV2) [dotted] -- (label2R);
    \draw [<->] (label2L) -- (label2R) node [midway, left] (TextNode) {semi-minor axis};

  \end{tikzpicture}
  \caption{Ellipse dimensions}
\end{figure}

The two foci are marked as $F_1$ and $F_2$. The two points at the ends of the majoraxis ($V_1$ and $V_2$) are sometimes refeerred to as ``vertices'' while the two points at the ends of the minor axis ($V_3$ and $V_4$) are referred to as ``co-vertices''.

\begin{tabular}{|l|l|l|p{0.4\linewidth}|}
  \hline
  $CF_1=CF_2$ & linear eccentricity & $c$ & half-focal separation ($e$, $f$)\\
$CV_1=CV_2$ & semi-major axis & $a$ \\
$CV_3=CV_4$ & semi-minor axis & $b$ \\
$\frac{c}{a}$ & eccentricity & $e$ & first eccentricity, mathematical eccentricity ($\epsilon$) \\
\hline 
\end{tabular}

$$e=\sqrt{\frac{a^2 - b^2}{a^2}}$$
$$c=\sqrt{a^2 - b^2}$$

\begin{figure}
  \begin{tikzpicture}
    \coordinate (pointC) at (0, 0);
    \fill (pointC) circle [radius=2pt];
    \draw (pointC) node [below left] {C};

    % Focus
    \coordinate (pointF1) at (4, 0);
    \fill (pointF1) circle [radius=2pt];
    \draw (pointF1) node [below] {$F_1$};

    % Vertex
    \coordinate (pointV1) at (5, 0);
    \fill (pointV1) circle [radius=2pt];
    \draw (pointV1) node [right] {$V$};


    % Ellipse
    \draw (pointC) [red] ellipse (5 and 3);


    % Axes
    \draw (pointV1) {[dotted] -- (pointF1)} -- (pointC) -- (0, 3);
    \draw (pointV2) [dotted] -- (pointC) -- (pointV4);


    % Points
    \coordinate (pointP) at (3, 2.4);
    \fill (pointP) circle [radius=2pt];
    \draw (pointP) node [above right] {P};

    \coordinate (pointPdash) at (3, 4);
    \fill (pointPdash) circle [radius=2pt];
    \draw (pointPdash) node [above right] {P};


    % Auxiliary circles
    \draw (pointC) [blue] circle (5);
    \draw (pointC) [green] circle (3);

    % Eccentric anomaly
    \draw (pointC) -- (pointPdash);
    \pic [draw, ->, "$\phi$", ultra thick, angle eccentricity=1.5, angle radius=1.5cm] {angle = pointV1--pointC--pointPdash};

    % True anomaly
    \draw (pointF1) -- (pointP);
    \pic [draw, ->, "$f$", ultra thick, angle eccentricity=1.5, angle radius=0.5cm] {angle = pointV1--pointF1--pointP};


  \end{tikzpicture}
  \caption{Ellipse angles}
  
\end{figure}

The {\emph true anomaly} of a point $P$ on an ellipse is the angle between the major axis and the line from a focus to that point -- $\angle VCP$ in the diagram labelled $c$. The true anomaly is also written $\theta$ or $\nu$ in the literature. 

The {\emph eccentric anomaly} is constructed by using an ``auxiliary'' circle of radius $a$. The point $P$ is projeted (?) ``up'' to the auxiliary circle onto point $P'$. The eccentric anomaly is then $\angle VCP'$, in other words the angle between the semi-major axis and the line connecting the centre to $P'$.

\subsection{The standard ellipse}
The standard ellipse is given by the cartesian coordinate formula:

$$ \frac{x^2}{a^2} + \frac{y^2}{b^2}=1$$

And has the following properties:

\begin{tabular}{ll}
  \text{Semi-major axis}  & $a$ \\
\text{Semi-minor axis} & $b$ \\
\text{Eccentricity} &  $\sqrt{\frac{a^2 - b^2}{a^2}}$ \\
\text{General point} & $(a cos(\phi), b sin(\phi))$ \\

\end{tabular}


\end{document}

\documentclass[main.tex]{subfiles}
\begin{document}
\section{The physics}
\subsection{Our scientists}
Set up situation.

Train drivers passing by each other. Really very fast trains. Actually had better be spaceships. With velocity $v$. Velocity could be measured in anything so we may need a conversion factor.

\subsection{Length contraction}
Each sees the other's measuring rods shrunk along the axis of movement by a factor of $\sqrt{1-\frac{v^2}{c^2}}$. This shrinkage factor is so important that it has its own name $\beta$ (it is also much easier to write $\beta$ in what follows. Here $c$ stands for a conversion factor $299792458ms^{-1}$, which makes $\frac{v^2}{c^2}$ dimensionless. It did not matter what we used. What $c$ means will make sense in a moment (forward ref).

(Fitzgerald contraction)

Note that because $\beta$ depends on $v^2$ it is the same for positive and negativ velocities. In other words both scientists see the same shrinkage of the other. You very occasionally hear it said ``rods contract if you move faster'' but that {\emph sounds} like it would be one-sided. How can both scientists think the other one's rod is contracting, who is right (neither?), what is going on. We aren't seeing the whole picture.

As velocity increases and gets closer to $c$ then $\beta$ tends to zero. (see graph). 

\begin{tikzpicture}
\begin{axis}[
    axis lines = left,
    xlabel = $v$,
    ylabel = {$\beta=\sqrt{1-v^2/c^2}$},
]
%Below the red parabola is defined
\addplot [
    domain=0:1, 
    samples=100, 
]
{sqrt(1 - x^2)};
\end{axis}
\end{tikzpicture}
\subsection{Time dilation}
In 1971, four atomic clocks were flown around the world: two Westwards and two Eastwards. On their return they were compared with two clocks that had been left at the US Naval Observatory. The Eastward clocks -- flying with the rotation of the Earth had lost about 59ns (59 x $10^-9$s). By contrast the Westward clocks -- flying against the rotation of the Earth had gained about 273 nanoseconds and so, on average, had run faster.

The second discrepancy the two scientists notice is one of time. 


\end{document}

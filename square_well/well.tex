\documentclass[a4paper, 12pt]{article}
\usepackage{mathtools}
\usepackage{bm}
\usepackage{tikz}
\usepackage{float}
\usepackage{pgfplots}
\setlength{\parindent}{0cm}
\setlength{\parskip}{12pt plus4mm minus3mm}

\usepgfplotslibrary{units}

\begin{document}
We consider a one-dimensional box, also known as an infinite square well. Intuitively, the particle is constrained to be in a box $[-L/2, L/2]$ of width $L$. In some treatments the box is $[0, L]$. The physics is the same, but the detailed algebra is different.

``Constrained'' could mean that when the particle hits the wall it is absorbed or reflected. There might be other possibilities since we are working in quatum mechanics where concepts such as ``reflected'' may be problematic. At this point many treatments introduce somewhat magical boundary conditions. 

\section{propagator}
To keep our assumptions clear, let us begin by solving the problem using path integrals. In this presentation, it is clear what ``reflected'' means for a path. Indeed we can make the mathematics much simpler by a ``method of images''.

Instead of a single square box, we would have an infinite numbers of copies of the box alternately reflected. The position $x$ in the box containing the origin would be mirrored by $L - x$ in the next box to the right and $-L -x$ to the left. Thus all the points $x-4L, x-2L, x, x+2L, x+4L$ and $-3L-x, -L-x, L-x, 3L-x$ are equivalent, representing a point reached by a different number of reflections in the walls.

This allows us to take the free particle Feynmann propagator:

$$K(x,t) = \frac{m}{2\pi i\hbar(t_b - t_a)}^{1/2}\exp{\frac{imx^2}{2\hbar t}}$$

and adapt it to the proapagator for the well $K_{well}$:

$$K_{WELL}(x, t) = \sum


\end{document}

\documentclass[a4paper, 12pt]{article}
\usepackage{mathtools}
\usepackage{bm}
\usepackage{tikz}
\usepackage{float}
\usepackage{pgfplots}
\setlength{\parindent}{0cm}
\setlength{\parskip}{12pt plus4mm minus3mm}

\usepgfplotslibrary{units}
\pgfplotsset{compat=1.17}

\usepackage{amsmath}
\usepackage{amsthm}

\theoremstyle{definition}
\newtheorem{definition}{Definition}[section]


\begin{document}
We consider a one-dimensional box, also known as an infinite square well. Intuitively, the particle is constrained to be in a box $[-L/2, L/2]$ of width $L$. In some treatments the box is $[0, L]$. The physics is the same, but the detailed algebra is different.

``Constrained'' could mean that when the particle hits the wall it is absorbed or reflected. There might be other possibilities since we are working in quatum mechanics where concepts such as ``reflected'' may be problematic. At this point many treatments introduce somewhat magical boundary conditions. 

\section{propagator}
To keep our assumptions clear, let us begin by solving the problem using path integrals. In this presentation, it is clear what ``reflected'' means for a path. Indeed we can make the mathematics much simpler by a ``method of images''.

Instead of a single square box, we would have an infinite numbers of copies of the box alternately reflected. The position $x$ in the box containing the origin would be mirrored by $L - x$ in the next box to the right and $-L -x$ to the left. Thus all the points $x-4L, x-2L, x, x+2L, x+4L$ and $-3L-x, -L-x, L-x, 3L-x$ are equivalent, representing a point reached by a different number of reflections in the walls.

This allows us to take the free particle Feynmann propagator:

$$K(x,t) = \frac{m}{2\pi i\hbar(t_b - t_a)}^{1/2}\exp{\frac{imx^2}{2\hbar t}}$$

and adapt it to the proapagator for the well $K_{well}$:

$$K_{WELL}(x, t) = \sum$$

\section{Operator formalism}
\begin{definition}[Hermitian Operator]
  An operator is {\emph Hermitian} if
  \begin{align*}
    <f|Lg> &= <Lf |g> \\
\int_{a}^{b}v^{*}Ludx &= \int_{a}^{b}(Lv)^{*}udx
  \end{align*}

\end{definition}
An alternative approach is first to define the {\emph Hermitian adjoint}. Given two vector spaces $H$ and $J$ equipped with inner products (?is this a true bilinear form or sesquilinear?), and a linear operator $L:H\rightarrow J$ then the adjoint of $L$ is an operator that maps from $J$ to $H$ with an additional restriction as follows:

Given $f$ in $H$ and $g$ in $J$, then:

\begin{align*}
  L:H&\rightarrow G \\
 H&\leftarrow G:L^{*}\\
L(f)\cdot g \text{ (in H)} &= f\cdot L^{*}(g) \text{ (in G)} 
\end{align*}

A Hermitian operator is then an operator taking that space to itself which is its own Hermitian adjoint. The inner product is usually not written this way in quantum mechanics, but the $A\cdot B$ notation is the least cluttered.

Being Hermitian on the kinds of Hilbert space used in quantum mechanics is affected by the way the operator behaves at boundaries. In other words it must be a lgobal property. There is a theory of extending operators to being Hermitian operators when they might otherwise fail.

For example in the square well as described the operator $\hat{p} = -i\hbar \partial$ is not Hermitian, but if we define $\Delta(x)=\delta(x + L) - \delta(x - L)$ then we get a Hermitian operator. See Daniel F. Styer - quantum revivals v classical, also Bonneau, Farant and Valent.

\begin{align*}
  -i\hbar\int_{-L}^{L}\phi^{*}(-i\hbar(\partial - \Delta)\psi)dx &= -\hbar^2 \int_{-L}^{L}\phi^{*}(\partial - \Delta)\psi)dx \\
& = -\hbar^2 \int_{-L}^{L}\phi^{*}\partial\psi dx - \int_{-L}^{L}\phi^{*}\Delta\psi dx \\
\intertext{Integrating the first term by parts and integrating out the delta functions in the second term}
& = \left[\phi^{*}\psi\right]_{-L}^{L} - \int_{-L}^{L}(\partial\phi^{*})\psi dx - \left[\phi^{*}\psi\right]_{-L}^{L} \\
\intertext{The first and last terms then cancel, leaving us with the desired result}

&=-i\hbar\int_{-L}^{L}(\partial\phi^{*})\psi dx
\end{align*}

\end{document}

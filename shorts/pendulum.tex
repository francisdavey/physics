\documentclass{article}
\usepackage{amsthm}
\usepackage{amsmath}
\usepackage{tikz}
\newtheorem{theorem}{Theorem}[section]
\begin{document}
Consider a simple pendulum of length $l$ at angle $\theta$ from the vertical (see illustration). The pendulum is massless, but has a mass $m$ at the end. We assume that we are working on scales at which the force due to gravity is a constant  $g$.  

\begin{figure}[h]

  \begin{tikzpicture}
    \coordinate (O) at (0,0);
    \fill (O) circle [radius=3pt];
    \draw (O) circle (4);
  \end{tikzpicture}
  
  \caption{The pendulum apparatus}
  
\end{figure}

\begin{theorem}
  \begin{align}
    E_{K} &= \frac{m}{2}l^2\dot{\theta}^2 \\
    E_{P} &= l(1 - cos(\theta))mg \\
\intertext{Equating the two gives us a differential equation for $\theta$}
\dot{\theta}^2 & =  \frac{g}{l}(1 - cos(\theta)) \\
\dot{\theta} &= \sqrt{\frac{g}{l}(1 - cos(\theta))}
  \end{align}
\end{theorem}
\end{document}
